%\documentclass[12pt,a4paper]{IEEEtran}
\documentclass[12pt,a4paper]{article}
\usepackage[utf8]{inputenc}
\usepackage[english]{babel}
\usepackage{amsmath}
\usepackage{amsfonts}
\usepackage{amssymb}
\usepackage{csquotes}
%\usepackage[left=2cm,right=2cm,top=2cm,bottom=2cm]{geometry}
\usepackage[backend=biber]{biblatex}
\bibliography{references}
\author{Schwarenthorer Yannick, 1229026}
%\title{\vspace{-3cm}Password meters}
\title{Password meters}

\begin{document}

\maketitle

\section*{Abstract}
\label{sec:abstract}
This summary is about password meters and the effectiveness of them. 
We will look into the details of how one of the currrent best meters (zxcvbn \cite{zxcvbn}) work and what charactaristics such a meter needs. The effect of the present of a meter of the password choosing behaviour \cite{upToEleven}. In the paper 

%\tableofcontents 


\section{Introduction}
\label{sec:introduction}
Passwords are still the main authentication mechanism on all kinds of systems and according to the leading tech guys they will be with us for at least 10 years.
The security and privacy of all our data relies on a human generated string of characters and numbers.
This rich structure makes them a target of guessing attacks.



\section{Comparison}
\label{sec:Comparison}
In this section we will compare some password meters from commonly know websites. We will explane briefly how all of them work internally.



\section{Zxcvbn}
\label{sec:zxcvbn}
Zxcvbn is the open sourced password meter developed from Dropbox inc. It checks for common keyboard patterns, and also performs a dictionary check with the rockYou password set.



\section{Effect of password meters}
\label{sec:effect}
In this section we will take a look at the effect of password meters.
In the paper \citetitle{upToEleven} \cite{upToEleven} the showed that they present of a password meter has an effect on the password strength. However this holds only for sites which the users rated as important. For lower risk websites (sites which store no personal information about the user) the password strength is not higher when a password meter is present.


\printbibliography


\end{document}
